\documentclass{IEEEtran}

\usepackage{amsmath,amssymb,amsthm}
\usepackage{bm}
\usepackage{graphicx}
\usepackage{multirow}
\usepackage{xcolor}
\usepackage{float}
\usepackage{tabularx}
\usepackage{algorithm}
\usepackage[noend]{algpseudocode}
\usepackage{cite}
\usepackage{caption}
\usepackage{subcaption}
\usepackage{placeins}
\usepackage{multicol}
\usepackage{cleveref}
\usepackage{flushend}

\crefformat{equation}{(#1)}
\crefformat{section}{#2Section~#1#3}
\crefformat{subsection}{#2Section~#1#3}
\crefformat{subsubsection}{#2Section~#1#3}
\crefformat{algorithm}{#2Algorithm~#1#3}
\crefformat{figure}{#2Fig.~#1#3}
\crefformat{table}{#2Table~#1#3}
\usepackage{graphicx}

\setlength\floatsep{1em plus 1pt minus 3pt}
\setlength\textfloatsep{1em plus 1pt minus 3pt}
\setlength\intextsep{1em plus 1pt minus 3 pt}


\begin{document}
\title{A Survey of LSH for Similarity Search}
\author{
\IEEEauthorblockN{Zeyan Li}
\IEEEauthorblockA{2018310816} \and
\IEEEauthorblockN{ChenCheng Xu}
\IEEEauthorblockA{2018310851}
}
\maketitle
\thispagestyle{plain}
\pagestyle{plain}

\begin{abstract}
  LSH is widely used in similarity search.
  We review some LSH schemes.
  We also implement some of them and compare their performance with MNIST dataset.
  We found that multi-probe LSH significantly reduce the number of hash table needed but requires more search time. LSH forest suffers from long build time.
\end{abstract}

\section{Introduction}
In past decades, locality sensitive hashing (LSH) has been widely proposed for similarity search in high-dimensional spaces, especially for content-based search of feature-rich data such as audio recordings, digital photos, digital videos, and other sensor data \cite{lv2007multi}. Generally speaking, there are two ways to apply hashing to similarity search \cite{wang2014hashing}. The first one is indexing data terms using hash tables, which is formed by storing the terms with the same code in hash buckets. In detail, algorithms in this way regard the items lying the buckets corresponding to the codes of the query as the similarity candidates and then determine nearest neighbors by checking the distance between the query and candidates. Most researches in this direction aim to design hash functions satisfying the locality sensitive property and to design efficient search schemes, such as Multi-probe LSH \cite{lv2007multi}, Entropy-based LSH \cite{panigrahy2006entropy} and query adaptative LSH \cite{jegou2008query}. The other way is to approximate the distance between instances by the one computed with hash code, which ranks the instances according to the distances computed using the short codes. Such algorithms include LSH forest \cite{bawa2005lsh} and dynamic collision counting based LSH \cite{gan2012locality}.

An ideal LSH scheme for similarity search should satisfy the following properties \cite{bawa2005lsh, lv2007multi}:

\begin{itemize}
	\item Accurate: $\frac{\#\ of\ True\ Near\ Neightbors}{\#\ of\ Retrieved\ Candidates}$ should be as large as possible.
	\item {Efficient Queries}: The number of retrieved candidates should be as small as possible to reduce I/O and computation costs.
	\item {Efficient Maintenance}: The index should be built in a single scan of the dataset, and subsequent inserts and deletes of objects should be efficient.
	\item  {Domain Independence}: The indexing scheme should adapt to any data domain and require no special tuning of parameters for each specific dataset.
	\item  {Minimum Storage}: Storage consumption should be as little as possible, ideally linear in the data size.
\end{itemize}
\section{Methods of LSH}

\subsection{Multi-Probe LSH}
Traditional LSH schemes directly return the objects in the exactly collision buckets.
Multi-probe LSH~\cite{lv2007multi} tries to probe more buckets in one hash table to reduce the number of hash tables while maintains similar performance.
Formally speaking, multi-probe LSH probes a sequence of buckets which are from the collision bucket and a sequence of perturbations, as \cref{tbl:multi-probe-lsh} shows.
\begin{table}[hbt]
\centering
\caption{Difference between basic LSH and multi-probe LSH}
\begin{tabular}{|c|c|}
\hline
  \textbf{Scheme} & \textbf{Query}\\ \hline
  Basic& $g(q)=(h_1(q), h_2(q), ..., h_M(q))$ \\ \hline
  Multi-Probe& $g(q){+}\Delta^{(i)}, i{=}1,2,...,T$, $\Delta^{(i)}{=}(\delta_1^{(i)}, \delta_2^{(i)}, ..., \delta_M^{(i)})$ \\ \hline
\end{tabular}
\label{tbl:multi-probe-lsh}
\end{table}

Then we are going to introduce how the perturbation sequences are constructed.
\subsubsection{Step-Wise Probing Sequence}
Given the properties of LSH, similar objects are more likely in the close buckets.
This motivates the step-wise probing sequence, which firstly probes all 1-step perturbations, then all 2-step perturbations, and so on.
There are $L{\times} {M\choose n} {\times} 2^{n}$ n-step perturbations in total, where $L$ denotes the number of hash tables and $M$ denotes the number of compound hash functions in each table.

\subsubsection{Query-Based Probing Sequence}
Step-wise probing just consider all coordinates to be equally likely.
However, according the properties LSH, some perturbations are more likely than others.

We consider this hash function: $h(q)=\lfloor\frac{a\cdot q + b}{w}\rfloor$, where $w$ is a fixed hyper-parameter, $a$ is drawn from standard Gaussian, and $b$ is uniformly drawn from $[0, w)$.
For two objects $p, q$, $f(p) - f(q)$ follows Gaussian Distribution, where $f(q){=}a\cdot q {+} b$.
Therefore $P(h(p)=h(q)+\Delta)\approx C \exp(\sum_{i=1}^{M} x_i(\delta_i)^2)$, where $x_i(\delta_i)$ is the distance between q's projection and the boundary, like \cref{fig:x_i} shows.

\begin{figure}[hbt]
\centering
  \includegraphics[width=\columnwidth]{figures/multi_probe_nn.png}
  \caption{$x_i(\delta)$}
  \label{fig:x_i}
\end{figure}

In order to find the probing sequence with smallest scores (defined as $score(\Delta)=\sum_{i=1}^{M}x_i(\delta_i)^2$), firstly we need to sort all $2M$ $x_i(\delta_i)$, then we use a heap to generate the probing sequence.

\subsubsection{Optimized Probing Sequence}

One of the drawback of query-based probing sequence is that it needs generating probing sequence for each query.
However, we can use an approximation to precompute it.
We denote the $j$-th elements in sorted $x_i(\delta_i)$ array  as $z_j$, then $z_j$ and $z_j^2$'s expectation is known and independent of query $q$.
Specifically speaking, $E[z_j]{=}\frac{j}{2(M+1)}W, E[z_j^2]{=}\frac{j(j+1)}{4(M+1)(M+2)}W^2$.
We use $\mathbb{E}[z_j^2]$ to replace $z_j^2$, then we only need to sort all $x_i(\delta_i)$ for each query.
\subsection{Dynamic Collision Counting LSH}

Traditional LSH schemes usually use static compound hash function to reduce false positives.
However, the static compound hash functions also reduce recall, and many hash tables have to be used to improve recall.

C2LSH~\cite{gan2012locality} uses dynamic compound hash function rather than a static one.
C2LSH firstly randomly chooses a set of $m$ LSDH functions with appropriately small interval $W$, which form a function base $\mathcal{B}$.
Only data objects with large enough collision counts need to have their distances computed.
A data object is called frequent if its collision number \#collision(o) is greater than or equal to a pre-specified collision threshold $l$.

C2LSH firstly calculate the buckets that $q$ falls in by $h_i(q), i=1,2,...,m$, and find the objects collides with $q$.

Then we compute \#collides(o) for every $o$ and hence identify the set $C$ of all frequent objects. Then we compute $max(\#C, \beta n)$ frequent members of C. $n$ is the the cardinality of the database.
The collision threshold is defined as $l=\alpha m$.
These two properties should hold to ensure C2LSH correct:
\begin{itemize}
	\item $\mathcal{P}_1$: If there exists a data object o, s.t. $o\in B(q, R)$, then o's collision number is at least l. 
	\item $\mathcal{P}_2$: The total number of false positives is less than $\beta n$.
\end{itemize}

In case of no data points returned, it uses virtual reranking to equivalently search a neighbor with radius $1, c, c^2, ...$.
\section{Evaluation}
\subsection{Datasets}
\subsection{Evaluation Metrics}
\subsection{Effectiveness}
\subsection{Time Efficiency}
\subsection{Space Efficiency}
\section{Conclusion}

LSH is widely used in similarity search and many LSH schemes are proposed.
We review some LSH schemes, and they all use different ways to reduce the consumption of LSH.

We also implement and compare several of them.
We found that multi-probe LSH significantly reduce the number of hash table needed.
But with respect to search time, it requires more search when achieving the same recall with LSH forest.
However, LSH forest suffers from huge build time.

\clearpage
\bibliographystyle{IEEEtran}
\bibliography{refs/refs.bib}
\end{document}