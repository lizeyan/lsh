\subsection{Dynamic Collision Counting LSH}

Traditional LSH schemes usually use static compound hash function to reduce false positives.
However, the static compound hash functions also reduce recall, and many hash tables have to be used to improve recall.

C2LSH~\cite{gan2012locality} uses dynamic compound hash function rather than a static one.
C2LSH firstly randomly chooses a set of $m$ LSDH functions with appropriately small interval $W$, which form a function base $\mathcal{B}$.
Only data objects with large enough collision counts need to have their distances computed.
A data object is called frequent if its collision number \#collision(o) is greater than or equal to a pre-specified collision threshold $l$.

C2LSH firstly calculate the buckets that $q$ falls in by $h_i(q), i=1,2,...,m$, and find the objects collides with $q$.

Then we compute \#collides(o) for every $o$ and hence identify the set $C$ of all frequent objects. Then we compute $max(\#C, \beta n)$ frequent members of C. $n$ is the the cardinality of the database.
The collision threshold is defined as $l=\alpha m$.
These two properties should hold to ensure C2LSH correct:
\begin{itemize}
	\item $\mathcal{P}_1$: If there exists a data object o, s.t. $o\in B(q, R)$, then o's collision number is at least l. 
	\item $\mathcal{P}_2$: The total number of false positives is less than $\beta n$.
\end{itemize}

In case of no data points returned, it uses virtual reranking to equivalently search a neighbor with radius $1, c, c^2, ...$.